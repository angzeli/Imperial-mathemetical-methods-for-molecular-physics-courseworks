\documentclass[a4paper,11pt]{article}
\usepackage{amsmath, amssymb, amsthm, physics}
\usepackage{graphicx}
\usepackage{hyperref}
\usepackage{geometry}
\geometry{left=1in, right=1in, top=1in, bottom=1in}
\setlength{\parindent}{0pt}
\setcounter{secnumdepth}{0}
\usepackage{mathtools}
\mathtoolsset{showonlyrefs}
\usepackage{caption}

\usepackage{xcolor}
\usepackage{fancyhdr}
\pagestyle{fancy}

\fancyhf{} % clear default header/footer
\fancyhead[L]{\textcolor{gray}{\small Angze Li \quad CID: 02373494}}
\fancyfoot[C]{\textcolor{gray}{\small \thepage}}

\renewcommand{\footrulewidth}{0pt}  % no line above the footer (optional)

\renewcommand{\headrulewidth}{0pt}  % no line under the header (optional)

% Title Page
\title{Mathematical Methods for Molecular Physics Coursework \#1} % Replace X with sheet number
\author{Angze Li \\ CID: 02373494 \\ Imperial College London}
\date{\today}

\begin{document}
\maketitle

\tableofcontents % Optional


\newpage

% Example Problem Format
\section{Problem 1}
\subsection*{Question}\textit{\textbf{Polynomial expansions}\\
1.1. To the accuracy of the first three non-vanishing terms, write down the Maclaurin expansion of the functions (near the point $x=0$):\\
\hspace*{2em} a) $e^{-\alpha x^2}$, b) $\ln(1+\alpha x^2)$, c) $(1+\alpha\sqrt x)^{\mu}$\\
1.2. Plot separately each function against its polynomial approximation for $\alpha=1$ and $\alpha=-1$, $µ=0.5$ and $µ=2$.
Plot a) and b) in the interval of $0\le x\le 2$ and plot c) in the interval $0 \le x \le 1$ \\
1.3. Using the plots from part 1.2, comment on where the polynomial approximations of a), b) and c) are good and where they stop reproducing the given functions.}

\subsection*{Solution}
\textbf{1.1} Since all three functions are to be approximated near $x=0$ and are analytic in a neighbourhood of the origin (for suitable $\alpha$), it is natural to use the Maclaurin expansion, which has the form
$$
f(x)=\sum_{n=0}^{+\infty}\frac{f^{(n)}(0)}{n!}x^n.
$$

\textbf{a)} For $e^{-\alpha x^2}$, it is convenient to introduce the substitution $t=-\alpha x^2$, so that the function becomes $f(t)=e^{t}$.
This reduces the problem to the well–known Maclaurin series of the exponential, which can be written without recomputing derivatives:
\begin{align}
f(t)
&= \frac{e^{t}\big|_{t=0}}{0!}t^0+\frac{e^t\big|_{t=0}}{1!}t^1+\frac{e^t\big|_{t=0}}{2!}t^2+\mathcal{O}(t^3)\nonumber\\
&= 1+t+\frac{1}{2}t^2+\mathcal{O}(t^3).\nonumber
\end{align}
Since we only need the first three non-vanishing terms and $t=-\alpha x^2$ is small whenever $x$ is small, truncating after $t^2$ is justified for $|x|\ll1$.
Therefore,
$$
f(x)=1-\alpha x^2+\frac{1}{2}\alpha^2x^4+\mathcal{O}(x^6).
$$

\textbf{b)} For $\ln(1+\alpha x^2)$ we again set $t=\alpha x^2$, so that $f(t)=\ln(1+t)$. This choice isolates the small parameter $t$ and allows us to use the standard Taylor series of $\ln(1+t)$ about $t=0$, which converges for $|t|<1$ (i.e. $|\alpha x^2|<1$):
\begin{align}
f(t)
&= \frac{\ln(1+t)\big|_{t=0}}{0!}t^0
+ \frac{\frac{1}{1+t}\big|_{t=0}}{1!}t^1
+ \frac{-\frac{1}{(1+t)^2}\big|_{t=0}}{2!}t^2
+ \frac{\frac{2}{(1+t)^3}\big|_{t=0}}{3!}t^3
+ \mathcal{O}(t^4)\nonumber\\
&= t-\frac{1}{2}t^2+\frac{1}{3}t^3+\mathcal{O}(t^4).\nonumber
\end{align}
Keeping only the first three non-vanishing terms is consistent with the required accuracy and is valid as long as $|\alpha x^2|\ll1$.
Substituting back $t=\alpha x^2$ gives
$$
f(x)=\alpha x^2-\frac{1}{2}\alpha^2x^4+\frac{1}{3}\alpha^3x^6+\mathcal{O}(x^8).
$$

\textbf{c)} For $(1+\alpha\sqrt{x})^{\mu}$, we set $t=\alpha\sqrt{x}$, so that $f(t)=(1+t)^{\mu}$. This allows us to use the generalized binomial expansion around $t=0$, which is appropriate because the problem asks for behaviour near $x=0$, i.e. $|\alpha\sqrt{x}| \ll 1$:
\begin{align}
f(t)
&= \frac{(1+t)^\mu\big|_{t=0}}{0!}t^0
+ \frac{\mu(1+t)^{\mu-1}\big|_{t=0}}{1!}t^1
+ \frac{\mu(\mu-1)(1+t)^{\mu-2}\big|_{t=0}}{2!}t^2
+\mathcal{O}(t^3)\nonumber\\
&= 1+\mu t+\frac{\mu(\mu-1)}{2}t^2+\mathcal{O}(t^3).\nonumber
\end{align}
Truncating after the $t^2$–term gives the first three non-vanishing contributions and is valid provided $|\alpha\sqrt{x}|$ remains small.
Substituting $t=\alpha\sqrt{x}$ yields
$$
f(x)=1+\mu\alpha \sqrt{x}+\frac{\mu(\mu-1)}{2}\alpha^2 x+\mathcal{O}(x^{3/2}).
$$
In all cases, the truncation is justified because we expand around $x=0$ and keep only the lowest powers of $x$ that contribute.\\
\textbf{1.2} 
\begin{figure}[ht]
    \centering
    \includegraphics[width=1\linewidth]{Figure_1.2_All_Functions_with_Params.png}
    \caption{Function Plots. Solid lines: the original function. Dashed lines: the approximated function.}
\end{figure}

\textbf{1.3} From the plots, the Maclaurin (polynomial) approximations reproduce the given functions accurately only for small values of $x$ (i.e. near $x=0$), where higher-order terms such as $x^4$ or $x^6$ remain negligible. For $e^{-\alpha x^2}$, the truncated series follows the exact curve well up to about $x\approx1$, but diverges afterwards—the error grows rapidly for large x because the exponential decays or increases faster than the polynomial. For $\ln(1+\alpha x^2)$, the approximation is valid within $|\alpha x^2|<1$; beyond this region the series either diverges ($\alpha=-1$) or drifts from the true curve. For $(1+\alpha\sqrt{x})^{\mu}$, the polynomial is reliable only for small $x$ where $|\alpha\sqrt{x}|<1$; when $x$ approaches 1, the deviation becomes visible as higher-order terms in $\sqrt{x}$ become significant. However, when $\mu=2$, the binomial expansion terminates after a finite number of terms—$(1\pm\sqrt{x})^{2}=1\pm2\sqrt{x}+x$—so the Maclaurin polynomial exactly equals the true function, resulting in a perfect fit across the entire interval. In summary, the majority of the approximations work well near the expansion point but lose accuracy once $x$ is no longer small.


\newpage
% Next problem

\section{Problem 2}
\subsection*{Question}
\textit{\textbf{Sketching functions using your understanding of its asymptotic properties}\\
Sketch the graph of a Fermi-like-function,\\
$$F(x,\alpha)=\frac{1}{1+\exp\{\alpha(x-x_0)\}}$$
In the interval of $0\le x < +\infty$ as a function of $x$ for the case of $\alpha>1, \alpha x_0 \gg 1$:\\
\hspace*{2em} a) Approximate this function near the point $x_0$ (using Taylor expansion).\\
\hspace*{2em} b) What does $F(x,\alpha)$ simplify to, at $x\gg x_0$?\\
\hspace*{2em} c) Sketch the whole function.}

\subsection*{Solution}
\textbf{a)}  
To approximate $F(x,\alpha)$ near $x=x_0$, we use a Taylor expansion about the point where the function changes most rapidly.  
Since $F(x,\alpha)$ is a smooth sigmoid and $\alpha>1$, the steepest variation occurs at $x=x_0$, where the exponent $\alpha(x-x_0)$ vanishes.  
Expanding around this point gives an accurate local description of the transition region.

We first compute derivatives evaluated at $x=x_0$:
\begin{align}
F^{(1)}(x,\alpha)\Big|_{x=x_0}
&= \frac{\alpha e^{\alpha(x-x_0)}}{(1+e^{\alpha(x-x_0)})^2}
\;\Rightarrow\;
F'(x_0,\alpha)=-\frac{\alpha}{4}, \nonumber\\[6pt]
%
F^{(2)}(x,\alpha)\Big|_{x=x_0}
&= -\frac{\alpha^2 (e^{\alpha(x-x_0)}-1)e^{\alpha(x-x_0)}}
{(e^{\alpha(x-x_0)}+1)^3}
\;\Rightarrow\;
F''(x_0,\alpha)=0, \nonumber\\[6pt]
%
F^{(3)}(x,\alpha)\Big|_{x=x_0}
&= -\frac{\alpha^{3} e^{\alpha (x - x_0)}
\big(e^{2\alpha (x - x_0)} - 4 e^{\alpha (x - x_0)} + 1\big)}
{(e^{\alpha (x - x_0)} + 1)^{4}}
\;\Rightarrow\;
F'''(x_0,\alpha)=\frac{1}{8}\alpha^{3}.\nonumber
\end{align}

Because $F''(x_0)=0$, the Taylor expansion contains no quadratic term, making the cubic term the next non-vanishing contribution.  
This is typical for symmetric sigmoidal functions expanded at their mid-point.

Keeping only the first three non-vanishing terms gives:
\[
F(x,\alpha) \approx
F(x_0) + F'(x_0)(x-x_0) + \frac{1}{6}F^{(3)}(x_0)(x-x_0)^3.
\]

Substituting the evaluated derivatives:
\[
F(x,\alpha)=\frac{1}{2}
-\frac{1}{4}\alpha(x-x_0)
+ \frac{1}{48}\alpha^{3}(x-x_0)^3.
\]

This truncation is justified because we are approximating the function only in a small neighbourhood of $x=x_0$, where higher-order terms in $(x-x_0)$ become negligible, and because $\alpha x_0\gg1$ ensures that the steep transition occurs sharply around this point.
\\

\textbf{b)} At $x\gg x_0$, the function can be reduced to:
$$\lim_{x\gg x_0}F(x,\alpha)=\frac{1}{1+e^{\alpha (x-x_0)}}\sim e^{-\alpha (x-x_0)},$$
since $e^{\alpha (x-x_0)}\gg 1$ at $x\gg x_0$.\\
\newpage
\textbf{c)}
\begin{figure}[h]
    \centering
    \includegraphics[width=0.5\linewidth]{Figure 2.1.png}
    \captionsetup{width=0.5\linewidth}
    \caption{Sketch of the Fermi-like function.}
\end{figure}

\newpage
\section{Problem 3}
\subsection*{Question} \textit{\textbf{Approximate solution of transcendental equations}\\
A popular equation appearing in the mean-field theory of critical phenomena has the form: $\alpha x=\tanh(x)$. Its solution(s) determines $x$ as a function of $\alpha$.\\
\hspace*{2em}a) How many solutions this equation has?
Under which circumstances this equation has only one solution?\\
\hspace*{2em}b) Find all approximate solutions, i.e. expression for, $x(\alpha)$, when $0<\alpha \ll 1$.\\
\hspace*{2em}c) Find all approximate solutions, i.e. expression for, $x(\alpha)$, when $\alpha<1,\ \alpha\to1$.
}
\subsection*{Solution}
\textbf{a)} The first derivative of $f(x)=\tanh(x)-\alpha x$ is:
$$f'(x) = \text{sech}^2(x)-\alpha.$$
i) For $\alpha\ge1$, since $\forall x\in R,\ \text{sech}^{2}x\le 1$,
$$f’(x)=\text{sech}^{2}x-\alpha\le 1-\alpha\le 0. $$
Hence $f(x)$ is a non-increasing function (flat when $\alpha=1$), crossing 0 at $x=0$. There is only one solution, which is $x=0$.\\
ii) For $\alpha\le0$, 
$$f’(x)=\text{sech}^{2}x-\alpha\ge -\alpha\ge0.$$
Hence the function is monotonically increasing function, crossing 0 at $x=0$. There is only one solution, which is $x=0$.\\
iii) For $0<\alpha<1$, when near 0, $\tanh(x)$ can be expanded into:
$$\tanh(x)=x-\frac{1}{3}x^3+\frac{2}{15}x^5+\mathcal{O}(x^7).$$
Hence,
$$f(x)=(1-\alpha)x-\frac{1}{3}x^3+\mathcal{O}(x^5).$$
$f(x)$ is therefore greater than 0 for a small value of $x$, and $f(x)=-\infty$ as $x\to+\infty$. Since the function is smooth, there will be a positive root $x^*>0$. By odd symmetry of $f(x)$, $-x^*$ is also a solution to the equation. The function will also have a solution at $x=0$. Therefore, there are three solutions.\\

\textbf{b)} When $0 < \alpha \ll 1$, the line $y = \alpha x$ has a very small slope, while the curve $y = \tanh x$ rises rapidly near $x=0$ and quickly saturates to $y=\pm1$.
Hence, the intersections occur near $x=0$ and at points where $\tanh x \to \pm 1$.

For large positive $x$, $\tanh x$ can be approximated by expanding in powers of $e^{-2x}$:
$$\tanh x = \frac{1 - e^{-2x}}{1 + e^{-2x}}
= 1 - 2e^{-2x} + 2e^{-4x} - 2e^{-6x} + ...$$
Since for large $x$, $e^{-2x}\ll1$, the higher-order terms are negligible and we can keep only the leading two terms:
$$\tanh x \approx 1 - 2e^{-2x}.$$

Substituting this into the original equation $\alpha x = \tanh x$ gives:
$$\alpha x = 1 - 2e^{-2x}.$$
Rearranging:
$$e^{-2x} = \frac{1 - \alpha x}{2}.$$
For small $\alpha$, the intersection occurs at large $x$, so we assume $x\approx 1/\alpha$ and add a small correction $s$, i.e.
$x = \frac{1}{\alpha} - s, \ \text{where}\ s \ll \frac{1}{\alpha}$.
\\Substituting this into the prior equation gives:
$$e^{-2(1/\alpha - s)} = \frac{1 - \alpha(1/\alpha - s)}{2}
\;\Rightarrow\;
e^{-2/\alpha}e^{2s} = \frac{\alpha s}{2}.$$

Multiply both sides by $\tfrac{2}{\alpha}e^{-2s}$:
$$\frac{2}{\alpha} e^{-2/\alpha} = s e^{-2s}.$$
Multiply by -2 to match the standard Lambert W form $u e^{u} = z$:
$$(-2s) e^{-2s} = -\frac{4}{\alpha} e^{-2/\alpha}.$$
Then, by definition of W(z):
$$-2s = W\left(-\frac{4}{\alpha} e^{-2/\alpha}\right)\Rightarrow 
s = -\frac{1}{2} W\!\left(-\frac{4}{\alpha} e^{-2/\alpha}\right).$$
Substituting $x = \tfrac{1}{\alpha} - s$, we obtain
$$x_+(\alpha)
= \frac{1}{\alpha} + \frac{1}{2}W\!\left(-\frac{4}{\alpha} e^{-2/\alpha}\right).$$
However, since the argument of W is exponentially small when $\alpha\ll1$, we can use the leading approximation $W(z)\approx z$, yielding
$$x_+(\alpha) \approx \frac{1}{\alpha} - \frac{2}{\alpha}e^{-2/\alpha}.$$
Because the full equation $\alpha x = \tanh x$ is odd, the negative solution is simply
$x_-(\alpha) = -x_+(\alpha)$,
and together with the trivial intersection at $x=0$, the approximate set of solutions is
$$x(\alpha) = 0,\;\pm\!\left(\frac{1}{\alpha} - \frac{2}{\alpha}e^{-2/\alpha}\right).$$


\textbf{c)} We can approximate $\tanh x$  by its Taylor series around $x=0$ in this regime where the non-zero solutions are expected to be small when $\alpha<1$ and $\alpha\to 1$, i.e.\ close to the critical value where the non-trivial solutions merge into $x=0$:
$$\alpha x= x-\frac{1}{3}x^3+\frac{2}{15}x^5...$$
Substituting this into the equation $\alpha x = \tanh x$ and keeping the first three non-vanishing terms gives
$$(1-\alpha)x-\frac{1}{3}x^3+\frac{2}{15}x^5=0.$$
We can introduce a small parameter $\varepsilon=1-\alpha\to0$ and multiply by 15 to give:
$$15\varepsilon x-5x^3+2x^5=0.$$
One obvious solution is $x=0$, which corresponds to the central intersection of the curves. To find the non-trivial solutions, we assume $x\neq 0$ and divide by $x$:
$$2t^2-5t+15\varepsilon=0.$$
The solutions of the quadratic equation are:
$$t=\frac{5\pm\sqrt{25-120\varepsilon}}{4}.$$
Since $t=x^{2}$, the corresponding solutions for the original variable are:
$$x=\sqrt{\frac{5\pm\sqrt{25-120(1-\alpha)}}{4}}.$$
However, the Taylor expansion of $\tanh x$ is valid only for sufficiently small $|x|$.  
For $\varepsilon \ll 1$, the ``$+$'' root in $t$ is of order $1$ and leads to $|x|$ that is not small, violating the small-$x$ assumption used in the expansion.  
The ``$-$'' root, on the other hand, tends to zero as $\varepsilon \to 0$ and thus remains within the regime where the truncated Taylor series is accurate.  

Therefore, within the validity of the approximation, the physically relevant non-zero solutions are given by
\[
x \approx \pm \sqrt{\frac{5 - \sqrt{25 - 120(1-\alpha)}}{4}},
\]
together with the trivial solution $x=0$.
\newpage
\section{Problem 4}
\subsection*{Question} \textit{\textbf{Integrals as functions}\\
4.1 Function $F(p)$ is defined as an integral: $F(p)=\int^{\infty}_0dx\frac{\exp(-px^2)}{1+x}.$\\
\hspace*{2em}a) To the accuracy of two non-vanishing terms, approximate this function at $p\gg1$.\\
\hspace*{2em}b) Plot numerically $F(p)$ (using Excel for the numerical calculation of integrals) and the approximate
result, comment on where the approximation is valid and where it breaks down.\\
4.2 Function $f(p)$ is defined as an integral: $f(p)=\int^{p}_0 \frac{\sin (kx)}{x}dx.$\\
\hspace*{2em}a) Find its small $p$-expansion to the accuracy of the first two non-vanishing terms.\\
\hspace*{2em}b) Put $k = 1$ and plot $f(p,1)$ numerically; compare it with the approximation obtained. Comment on where the approximation is valid and where it obviously breaks down.
}
\subsection*{Solution}
\textbf{4.1}\\
\textbf{a)} The Maclaurin expansion of $(1+x)^{-1}$ is used here since, for $p\gg 1$, the integral $F(p)$ is dominated by the region of \emph{small} $x$. And when $p$ is large, the factor $e^{-px^{2}}$ decays extremely rapidly, hence the main contribution to the integral is in the region where $x\ll 1$, so it is therefore justified to replace
\begin{align} (1+x)^{-1}
&=  1 + \frac{-1}{1!}x + \frac{-1(-1-1)}{2!}x^2 + \frac{-1(-1-1)(-1-2)}{3!}x^3 + ...\nonumber\\
&= 1-x+x^2-x^3+...
\nonumber\end{align}
by the first two non-vanishing terms:
$$(1+x)^{-1}\approx1-x.$$
Thus the integral is:

\begin{align} F(p)
&=  \int^{\infty}_0dx\frac{\exp(-px^2)}{1+x} \nonumber\\
&= \int^{\infty}_0dx\ \exp(-px^2)(1-x) \nonumber\\
&= \underbrace{\int^{\infty}_0dx\ \exp(-px^2)}_{I_1}-\underbrace{\int^{\infty}_0dx\ x\exp(-px^2)}_{I_2}.
\nonumber\end{align}
For $I_1$, let $y=\sqrt p x\Rightarrow dy=\sqrt p\ dx$. Substitution gives:
$$I_1 = \int_0^{\infty}dy\  \frac{e^{-y^2}}{\sqrt{p}}
= \frac{1}{\sqrt{p}} \cdot \frac{\sqrt{\pi}}{2}.$$
For $I_2$, let $u=-px^2\Rightarrow du=2px\ dx$. Substitution gives: 
$$I_2 = \int_0^{\infty} e^{-u}\frac{du}{2p}
= \frac{1}{2p}.$$
Hence,
$$F(p)=I_1-I_2=\frac{\sqrt\pi}{2\sqrt p }-\frac{1}{2p}.$$
\newpage
\textbf{b)}
\begin{figure}[ht!]
    \centering
    \includegraphics[width=0.6\linewidth]{Fig 4.1 Comparison of numerical and approximate F(p).png}
    \captionsetup{width=0.6\linewidth}
    \caption{Comparison of numerical and approximate F(p). Solid lines: the original function. Dashed lines: the approximated function.}
\end{figure}

The approximate expression
$$F_{\text{approx}}(p) = \frac{\sqrt{\pi}}{2\sqrt{p}} - \frac{1}{2p}$$
is derived from expanding the integrand $(1+x)^{-1}$ as $1 - x + \cdots$, and is therefore valid when the integral is dominated by small $x$ values. This occurs for large $p$, where the exponential factor $e^{-p x^2}$ decays rapidly with increasing $x$, and the first two terms of the Taylor expansion provide an excellent approximation. Hence $F_{\text{approx}}(p)$ reproduces the numerical result almost exactly. As $p$ decreases, the Gaussian broadens and the contribution from larger $x$ becomes more significant. In this case, higher-order terms in the expansion of $(1+x)^{-1}$ are no longer negligible, and the two-term approximation begins to deviate from the numerical curve.
Specifically, for $p \lesssim 4, F_{\text{approx}}(p)$ starts to underestimate the true integral, and the discrepancy grows rapidly as p approaches zero.\\
\textbf{4.2}\\
\textbf{a)} We use the Maclaurin expansion of $\sin(kx)$ because, for small $p$, the upper integration limit in $f(p)$ ensures that the integrand is evaluated only for $0\le x\le p$, where $x\ll 1$.  
In this regime, the argument $kx$ is also small, and the Taylor series
$$\sin kx =\sum_{n=0}^{\infty}\frac{(-1)^n (kx)^{2n+1}}{(2n+1)!}=kx-\frac{(kx)^3}{3!}+\frac{(kx)^5}{5!}-\frac{(kx)^7}{7!}+...$$
converges rapidly, where we can only retain the first two non-vanishing terms 
$$kx-\frac{(kx)^3}{3!}.$$
Therefore,
\begin{align} f(p,k)
&=  \int^p_0\frac{\sin kx}{x} \ dx\nonumber\\
&= \int^p_0\frac{kx-\frac{(kx)^3}{3!}}{x}\ dx \nonumber\\
&= \int^p_0k-\frac{k^3x^2}{6}\ dx
\nonumber\\
&= \Big[kx-\frac{k^3x^3}{18} \Big]^p_0\nonumber\\
&= kp-\frac{k^3p^3}{18}.\nonumber
\end{align}
\textbf{b)} Let $k=1$, then
$$f(p,1)=\int^p_0\frac{\sin x}{x}\ dx.$$
\begin{figure}[h]
    \centering
    \includegraphics[width=0.6\linewidth]{Fig 4.2 Comparison of numerical and approximate f(p).png}
    \captionsetup{width=0.6\linewidth}
    \caption{Comparison of numerical and approximate f(p). Solid line: the original function. Dashed line: the approximated function.}
\end{figure}

The approximate expression
$$f_{\text{approx}}(p) = p - \frac{p^{3}}{18}$$
is derived from the Maclaurin expansion of $\sin x$ and is therefore accurate only for small arguments (i.e., $p \ll 1$, where $\frac{\sin x}{x} \approx 1 - \frac{x^2}{6})$.

In the plot, the approximation (orange) agrees closely with the numerical result (blue) up to about $p \approx 1.5$, where the cubic term still captures the curvature of the true function. Beyond this region, the approximation begins to deviate: it starts to underestimate $f(p,1)$ after the peak and eventually decreases rapidly, while the true function slowly approaches the asymptotic value $\frac{\pi}{2}$. This breakdown occurs because higher-order terms in the series become non-negligible, and a low-order polynomial cannot reproduce the oscillatory or saturating behaviour of the sine integral at large $p$.

\newpage
\section{Problem 5}
\subsection*{Question} \textit{\textbf{Using approximations to answer physical questions}\\
The Lennard-Jones potential $U(R)=U_0[(\frac{a}{R})^{12}-(\frac{a}{R})^{6}]$\\ Approximates interactions of neutral species, most suitable for atoms of noble gases, where $R$ is the distance between
the atomic centres, $a$ is the characteristic size of atoms, $U_0$ is the constant determining the energy scale. For Ar, $U_0\approx 0.08 \ \text{eV},\ a=0.36\ \text{nm}$. \\
\hspace*{2em}5.1. Explain the meaning of the two terms in this equation, which effects they represent?\\
\hspace*{2em}5.2. Find an expression for the vibration frequency of Ar$_2$ dimer that may be kept by this potential.\\
\hspace*{2em}5.3. Compare the result with the vibration frequency of the H$_2$ molecule (which you should estimate following the
corresponding pages of the handouts of Lecture 2).\\
\hspace*{2em}5.4 Estimate for which temperatures the Ar$_2$ dimer will dissociate? Explain why at room temperature there is no such
thing as an Ar$_2$ molecule. 
}
\subsection*{Solution}
\textbf{5.1} The potential energy function consists of two competing terms with distinct physical meanings:\\
The first term, $\displaystyle U_0\left(\frac{a}{R}\right)^{12}$, represents the short-range repulsive interaction, which becomes dominant when the atoms are very close to each other ($R<a$).
It originates from the Pauli exclusion principle, preventing the overlap of electron clouds belonging to different atoms.\\

The second term, $\displaystyle -U_0\left(\frac{a}{R}\right)^{6}$, corresponds to the long-range attractive interaction. It arises from the London dispersion forces (induced dipole–induced dipole interactions) between neutral atoms. The $R^{-6}$ dependence comes from quantum mechanical treatment of these van der Waals forces.\\

The competition between the attractive and repulsive terms gives rise to a potential minimum at an equilibrium distance where the net force between the atoms is zero and the configuration is most stable.\\

\textbf{5.2} The first derivative of the Lennard-Jones potential can be expressed as:
$$\frac{dU}{dR}=U_0\Big[-12a^{12}R^{-13}+6a^6R^{-7} \Big].$$
Let $\frac{dU}{dR}=0$, the equilibrium distance $R_0$ can be solved as follows:
$$-12\cdot a^{12}R_0^{-13}+a^6R_0^{-7}=0\ \Rightarrow\ R_0=2^{1/6}a.$$
The second derivative of the Lennard Jones potential can be expressed as:
$$\frac{d^2U}{dR^2}=U_0\Big[156a^{12}R^{-14}-42a^6R^{-8} \Big].$$
The spring constant $k$ can be found as follows:
$$k=\frac{d^2U}{dR^2}=U_0\Big[156a^{12}R^{-14}-42a^6R^{-8} \Big].$$
Let $a^6R^{-6}=\frac{1}{2}$, the spring constant is:
$$k=U_0\Big[156a^{12}(2^{1/6}a)^{-14}-42a^6(2^{1/6}a)^{-8} \Big]\ \Rightarrow\ k=18\cdot 2^{-1/3}U_0a^{-2}.$$
Since the reduced mass of Ar$_2$ is:$\mu_{\text{Ar}_2}=\frac{m_{\text{Ar}}}{2}=20m_p$, the vibrational frequency of Ar$_2$ is:
\begin{align} \omega
&= \sqrt{\frac{k}{m}} \nonumber\\
&=  \sqrt{\frac{18\cdot2^{-1/3}U_0a^{-2}}{20m_p}}\nonumber\\
&= \sqrt{\frac{9\cdot2^{-1/3}U_0}{10m_pa^2}}.\nonumber
\end{align}
Using the parameters, the vibration frequency is:
\begin{align} \omega
&= \sqrt{\frac{9\cdot2^{-1/3}\cdot 0.08\cdot 1.6\times10^{-19}\ \text{J}}{10\cdot 1.67377\times 10^{-27}\ \text{kg}\ \cdot (0.36\times 10^{-9}\ \text{m})^2}}\nonumber\\
&= 6.49\times10^{12}\ \text{s$^{-1}$}\ (\text{to 3 s.f.}).\nonumber
\end{align}
\textbf{5.3} The vibrational frequency of H$_2$ given in Lecture 2 is $8.3\times10^{14}$ s$^{-1}$, which is higher than that of the supposed Ar$_2$ in two orders of magnitude. \\
This difference is due to the fact that H$_2$ has a strong covalent bond (i.e., large force constant $k\sim 500\text{–}600\ \mathrm{N\,m^{-1}}$) and a very small reduced mass $\mu_{\mathrm{H_2}}=m_{\mathrm{H}}/2$. On the contrary, along with a much greater reduced mass, Ar$_2$ is only very weakly bound by Van der Waals forces, leading to a very long bond distance and low bond strength, hence much lower vibrational frequency.\\
\textbf{5.4} The dissociation temperature $T_{\text{diss}}$ can be estimated by the formula below:
$$k_B T_{\text{diss}} \sim D_0.$$
The lowest dissociation energy $D_0$ can be calculated as:
\begin{align} D_0
&= E_{\text{free atoms}}-E_{v=0}\nonumber\\
&= \varepsilon - \tfrac12\hbar\omega.\nonumber
\end{align}
Using $\varepsilon=U_0/4=0.02\ \text{eV}$, $D_0$ is:
\begin{align} D_0
&= 0.02\cdot 1.6\times10^{-19}\ \text{J}-0.5\cdot1.055\times10^{-34}\ \text{J}\cdot\text{s}\cdot 6.49\times10^{12}\ \text{s$^{-1}$}\nonumber\\
&= 2.86\times10^{-21}\ \text{J}\ (\text{to 3 s.f.}).\nonumber
\end{align}
Therefore,
$$T_{\text{diss}}\ge\frac{D_0}{k_B}=\frac{2.86\times10^{-21}\ \text{J}}{1.38 \times 10^{-23}\ \text{J}\cdot \text{K}^{-1}}=207\ \text{K}\ (\text{to 3 s.f.}).$$
At room temperature (295 K), the thermal energy from random motion of Ar$_2$ would exceed the binding energy. Any transiently formed dimer will be immediately broken apart by collisions, leading to monoatomic structure.

\newpage
\section{Problem 6}
\subsection*{Question} \textit{\textbf{Functions of more than one variables}\\
6.1 The ideal gas equation $ pV=nRT$ relates pressure $p$, volume $V$, and absolute temperature, $T$ ($n$ is the number of
moles occupying the given volume, $R$ is the gas constant).\\
\hspace*{2em}a) Derive the expression for isothermal compressibility, $\frac{\partial V}{\partial p}\Big|_T$, for the ideal gas. Interpret its pressure-dependence.\\
\hspace*{2em}b) Derive the expression for isobaric thermal expansion coefficient, $\frac{\partial V}{\partial T}\Big|_p$ , for the ideal gas.\\
6.2. van der Waals equation (Johannes Diderik van der Waals, 1837-1923) extends the ideal gas equation to include attractive interaction between atoms and molecules (van der Waals forces) and the so called ‘excluded volume’ due to hard core repulsion between atoms/molecules due to collisions in compressed state. VdW-equation even describe in a qualitative way the transition to a liquid state at high pressures and reduced temperatures. If we are not going that far, it just corrects the ideal gas equation.
The VdW equation reads
$$\Big(p+\frac{n^2a}{V^2}\Big)(V-nb)=nRT$$
Here $a$ is a constant characterizing the attraction between atoms/molecules, and $b$ is the volume excluded by one mole of atoms/molecules due to their physical volume (so that
$V-nb$ is the ‘free volume of the space between the atoms/molecules) . The parameters $a$ and $b$ are generally substance dependent.\\
\hspace*{2em}a) Show in which limiting case in terms of volume the ideal gas equation is recovered. Show in which particular case
in terms of parameters $a$ and $b$ the ideal gas equation is recovered. Interpret your results.\\
\hspace*{2em}b) Find the parametric dependence of isothermal compressibility on pressure.\\
\hspace*{2em}c)  Find a saddle point on the surface defined by equation $S(x,y)=x^2-y^2$
}
\subsection*{Solution}
\textbf{6.1 a)} 
$$pV=nRT$$
$$\Rightarrow V=\frac{nRT}{p}$$
$$\Rightarrow \frac{\partial V}{\partial p}\Big|_{T}=-\frac{nRT}{p^2}=-\frac{V}{p}.$$
The rate of change of volume with pressure is inversely proportional to $p$, i.e., the volume becomes less sensitive to further increases in pressure.\\
\textbf{6.1 b)} 
$$pV=nRT$$
$$\Rightarrow V=\frac{nRT}{p}$$
$$\Rightarrow \frac{\partial V}{\partial T}\Big|_{p}=\frac{nR}{p}.$$

\textbf{6.2 a)} 
At very large V, both correction terms in the Van der Waals equation become negligible:
$$\frac{n^2a}{V^2} \to 0, \qquad nb \ll V.$$
Substituting these limits into
$$\Big(p+\frac{n^2a}{V^2}\Big)(V-nb)=nRT$$
gives
$$pV \approx nRT.$$
From a physical understanding, in the limit of large volume (low density), intermolecular attractions and excluded volume effects vanish, and the ideal gas law is recovered.\\
The ideal gas equation $pV=nRT$ is recovered for all $p,\ V,\ T$ if
$$a=0 \quad \text{and} \quad b=0.$$
From a physical understanding, $a=0$ means there is no attractive (van der Waals) forces between particles, and $b=0$ means particles have no finite size (no excluded volume), i.e. point particles. Together, this is exactly the ideal-gas model: non-interacting point particles.\\
\textbf{6.2 b)} 
$$\Big(p+\frac{n^2a}{V^2}\Big)(V-nb)=nRT$$
$$\Rightarrow p=\ \frac{nRT}{V-nb}-\frac{n^2a}{V^2}$$
\begin{align} \Rightarrow \frac{\partial p}{\partial V}\Big|_T
&= \frac{d}{dV}\Big(\frac{nRT}{V-nb}\Big)
-\frac{d}{dV}\Big(\frac{n^2a}{V^2}\Big) \nonumber\\
&= -\frac{nRT}{(V-nb)^2}+\frac{2n^2a}{V^3}. \nonumber 
\end{align}
$$\Rightarrow \frac{\partial V}{\partial p}\Big|_T=\frac{1}{\frac{\partial p}{\partial V}\Big|_T}=\frac{1}{-\frac{nRT}{(V-nb)^2}+\frac{2n^2a}{V^3}}.$$
\textbf{6.3}
\[
S(x,y)=x^2-y^2
\;\Rightarrow\;
\left\{
\begin{aligned}
    \frac{\partial S(x,y)}{\partial x} &= 2x\\
    \frac{\partial S(x,y)}{\partial y} &= -2y
\end{aligned}
\right.
\;\Rightarrow\;
\left\{
\begin{aligned}
    \frac{\partial^2 S(x,y)}{\partial x^2} &= 2>0\\
    \frac{\partial^2 S(x,y)}{\partial y^2} &= -2<0\\
    \frac{\partial^2 S(x,y)}{\partial x \partial y} &= 0
\end{aligned}
\right.
\]
where
$$\frac{\partial S(x,y)}{\partial x}=\frac{\partial S(x,y)}{\partial y}=0\; \text{at (0,0)}.$$
Therefore, 
$$S=\frac{\partial^2S(x,y)}{\partial x^2}\frac{\partial^2S(x,y)}{\partial y^2}-\Big(\frac{\partial^2S}{\partial x\partial y}\Big)^2=-4<0.$$
Since $S<0$, and 
$$\frac{\partial^2 S(x,y)}{\partial x^2}\cdot\frac{\partial^2 S(x,y)}{\partial y^2}<0,$$
it is determined that (0,0) is a saddle point on the surface.
\end{document}
