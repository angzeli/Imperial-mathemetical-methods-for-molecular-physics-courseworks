\documentclass[a4paper,11pt]{article}
\usepackage{amsmath, amssymb, amsthm, physics}
\usepackage{graphicx}
\usepackage{hyperref}
\usepackage{geometry}
\geometry{left=1in, right=1in, top=1in, bottom=1in}
\setlength{\parindent}{0pt}
\setcounter{secnumdepth}{0}
\usepackage{mathtools}
\mathtoolsset{showonlyrefs}
\usepackage{caption}

\usepackage{xcolor}
\usepackage{fancyhdr}
\pagestyle{fancy}

\fancyhf{} % clear default header/footer
\fancyhead[L]{\textcolor{gray}{\small Angze Li \quad CID: 02373494}}
\fancyfoot[C]{\textcolor{gray}{\small \thepage}}

\renewcommand{\footrulewidth}{0pt}  % no line above the footer (optional)

\renewcommand{\headrulewidth}{0pt}  % no line under the header (optional)

% Title Page
\title{Mathematical Methods for Molecular Physics Coursework \#2} 
\author{Angze Li \\ CID: 02373494 \\ Imperial College London}
\date{\today}

\begin{document}
\maketitle

\tableofcontents % Optional


\newpage

\section{1. Fourier analysis}

\subsection{1.1 Fourier frequency spectra}
\subsubsection*{a)}
\begin{align}
\tilde f(\omega)
&= \frac{1}{2\pi}  \int^{+\infty}_{-\infty} dt\ e^{-i\omega t} f(t)\\
&= \frac{1}{2\pi}  \int^{+\infty}_{-\infty} dt\ e^{-i\omega t} A\cos(\Omega t) \\
&= \frac{A}{2\pi} \int^{+\infty}_{-\infty} dt\ e^{-i\omega t}\Big(\frac{e^{i\Omega t}+e^{-i\Omega t}}{2}\Big)\\
&= \frac{A}{4\pi} \int^{+\infty}_{-\infty} dt\ \Big(e^{i(\omega-\Omega) t}+e^{i(-\omega-\Omega) t}\Big)\\
&= \frac{A}{4\pi} \Big(2\pi \delta{(\omega-\Omega) }+2\pi \delta{(-\omega-\Omega) }\Big)\\
&= \frac{A}{2} \Big( \delta{(\omega-\Omega) }+ \delta{(-\omega-\Omega) }\Big).
\end{align}
Since $\delta (x)=\delta(-x)$,
$$\tilde f(\omega)=\frac{A}{2} \Big( \delta{(\omega-\Omega) }+ \delta{(\omega+\Omega) }\Big).
$$
\subsubsection*{b)}

\begin{align}
\tilde f(\omega)
&= \frac{1}{2\pi} \int^{+\infty}_{-\infty} dt\ e^{-i\omega t} f(t)\\
&= \frac{1}{2\pi} \int^{+\infty}_{-\infty} dt\ e^{-i\omega t} Ae^{-t/\tau}\cos(\Omega t) \theta(t).
\end{align}
Since $\theta(t)=1$ at $t>0$ and 0 at $t<0$,
\begin{align}
\tilde f(\omega)
&= \frac{A}{2\pi} \int^{+\infty}_{0} dt\ e^{-i\omega t} e^{-t/\tau}\cos(\Omega t)\\
&= \frac{A}{2\pi} \int^{+\infty}_{0} dt\ e^{(-i\omega-\frac{1}{\tau}) t} \frac{e^{i\Omega t}+e^{-i\Omega t}}{2} \\
&= \frac{A}{4\pi} \int^{+\infty}_{0} dt\ e^{(i\Omega-i\omega-\frac{1}{\tau}) t}+e^{(-i\Omega-i\omega-\frac{1}{\tau}) t}.
\end{align}
Let $\alpha_1=[i(\Omega-\omega)-\frac{1}{\tau}]$ and $\alpha_2=[-i(\Omega+\omega)-\frac{1}{\tau}]$, then
\begin{align}
\tilde f(\omega)
&= \frac{A}{4\pi} \int^{+\infty}_{0} dt\ e^{\alpha_1 t}+e^{\alpha_2 t}\\
&= \frac{A}{4\pi} \Big[ \frac{e^{\alpha_1 t}}{\alpha_1}+\frac{e^{\alpha_2 t}}{\alpha_2}\Big]^{+\infty}_0
\end{align}
As the real part of $\alpha_1$ and $\alpha_2$ are negative, the integral is converged to 0:
$$\lim_{t\to+\infty}e^{\alpha t}=0.$$
\begin{align}
\tilde f(\omega)
&= \frac{A}{4\pi} \Big( -\frac{1}{\alpha_1}-\frac{1}{\alpha_2}\Big)\\
&= \frac{-A}{4\pi}\Big( \frac{1}{i(\Omega-\omega)-\frac{1}{\tau}}+\frac{1}{-i(\Omega+\omega)-\frac{1}{\tau}}\Big)\\
&= \frac{-A}{4\pi}\Big(\frac{-\frac{1}{\tau}}{(\frac{1}{\tau})^2+(\Omega-\omega)^2}-i\frac{\Omega-\omega}{(\frac{1}{\tau})^2+(\Omega-\omega)^2}+ \frac{-\frac{1}{\tau}}{(\frac{1}{\tau})^2+(\Omega+\omega)^2}-\frac{-(\Omega+\omega)}{(\frac{1}{\tau})^2+(\Omega+\omega)^2}\Big)\\
&= \frac{-A}{4\pi}\Big[-\big(\frac{\frac{1}{\tau}}{(\frac{1}{\tau})^2+(\Omega-\omega)^2}+\frac{\frac{1}{\tau}}{(\frac{1}{\tau})^2+(\Omega+\omega)^2}\big)+i\big(\frac{\Omega+\omega}{(\frac{1}{\tau})^2+(\Omega+\omega)^2}- \frac{\Omega-\omega}{(\frac{1}{\tau})^2+(\Omega-\omega)^2}\big)
\Big].
\end{align}
Therefore,
\begin{align}
\mathrm{Re} \big( \tilde f(\omega)\big)=\frac{A}{4\pi}\big(\frac{\tau}{1+(\Omega-\omega)^2\tau^2}+\frac{\tau}{1+(\Omega+\omega)^2\tau^2}\big),\\
\mathrm{Im} \big( \tilde f(\omega)\big)=\frac{A}{4\pi}\big( \frac{\Omega-\omega}{(\frac{1}{\tau})^2+(\Omega-\omega)^2}-\frac{\Omega+\omega}{(\frac{1}{\tau})^2+(\Omega+\omega)^2}\big).
\end{align}

\subsubsection*{c)}
We define the Lorentzian family (centred at $\omega_0$) as:
$$L_\tau(\omega-\omega_0)\equiv  \frac{1}{\pi}\frac{\tfrac{1}{\tau}}{(\omega-\omega_0)^2+(\tfrac{1}{\tau})^2}
=\frac{1}{\pi}\frac{\tau}{1+\tau^2(\omega-\omega_0)^2}.$$
Let $x=\tau(\omega-\omega_0)$, then $d\omega= dx/\tau$. It can be proved that this expression is normalised:
$$\int^{+\infty}_{-\infty}d\omega\ L_\tau(\omega-\omega_0)=\frac{1}{\pi}\int^{+\infty}_{-\infty}dx\frac{1}{1+x^2}=1$$
using 
$$\int^{+\infty}_{-\infty}dx\frac{1}{1+x^2}=\pi.$$
As $\tau\to +\infty$, the width of $L_\tau$ scales as $1/\tau\to 0$ while the area remains $1$. Therefore, $L_\tau$ is a Delta-like function, and
$$\lim_{\tau\to+\infty}L_\tau(\omega-\omega_0)\equiv\delta(\omega-\omega_0).$$
We can rewrite the real part of the result as:
$$\mathrm{Re} \big( \tilde f(\omega)\big)=\frac{A}{4}\big(L_\tau(\omega-\Omega)+L_\tau(\omega+\Omega) \big).$$
Therefore,
$$\lim_{\tau\to+\infty}\mathrm{Re} \big( \tilde f(\omega)\big)=\frac{A}{4}\big(\delta(\omega-\Omega)+\delta(\omega+\Omega) \big).$$
\newpage
\subsection{1.2 Fourier in 3D space}
\subsubsection*{a)}
\begin{align}
\tilde U(q) &= \int_0^{+\infty} d\mathbf{r}\ e^{-i\mathbf q\mathbf r}U(|\mathbf r|)\\
&= \int_0^{+\infty} dr\  r^2\int^{2\pi}_0d\phi\int ^\pi_0d\theta\ \sin\theta\cdot e^{-iqr\cos\theta}U(|\mathbf r|)\\
&= \int_0^{+\infty} dr\  r^2\cdot2\pi\cdot U(|\mathbf r|)\int ^\pi_0d\theta\ \sin\theta\cdot e^{-iqr\cos\theta}.
\end{align}
Let $t=\cos\theta$, then $dt=-\sin\theta$,
\begin{align}
\tilde U(q) &=  \int_0^{+\infty} dr\ 2\pi r^2U(|\mathbf r|)\int ^{-1}_1dt\ - e^{iqrt}\\
&= \int_0^{+\infty} dr\ 2\pi r^2U(|\mathbf r|)\cdot-\Big[\frac{1}{iqr}e^{iqrt} \Big]^{-1}_1 \\
&= \int_0^{+\infty} dr\ 2\pi r^2U(|\mathbf r|)\cdot-\frac{1}{iqr}\big(e^{-iqr}-e^{iqr}\big)\\
&= \int_0^{+\infty} dr\ 2\pi r^2U(|\mathbf r|)\cdot-\frac{1}{qr}2\sin(qr)\\
&= \frac{4\pi}{q}\int_0^{+\infty} dr\ r\sin(qr)\cdot U(|\mathbf r|).
\end{align}
Let the Yukawa-like potential be
$$U_\text{Y}(\mathbf{r})=k\frac{e^{-r/a}}{4\pi a^2r},$$
where $k$ is a constant. Then,
\begin{align}
\tilde U(q) 
&= \frac{4\pi}{q}\int_0^{+\infty} dr\ r\sin(qr)\cdot k\frac{e^{-r/a}}{4\pi a^2r}\\
&= \frac{k}{qa^2}\int_0^{+\infty} dr\ \sin(qr)e^{-r/a}.
\end{align}
Using the standard integral,
\begin{align}
\tilde U(q) 
&= \frac{k}{qa^2}\cdot \frac{aq}{1+a^2q^2}\\
&=k\frac{1}{a+a^3q^2}.
\end{align}
Therefore,
$$\tilde U(q)  \propto \frac{1}{a(1+a^2q^2)}.$$
Since $a$ is also a constant,
$$\tilde U(q)  \propto \frac{1}{1+a^2q^2}.$$
As $V_{si}\propto\tilde U(q)$ and $\sigma(\mathbf{q})\propto |V_{si}|^2$,
\begin{align}
V_{si}(\mathbf{q})\propto\frac{1}{1+a^2q^2},\\
\sigma(\mathbf{q})\propto \frac{1}{(1+a^2q^2)^2}.
\end{align}

\subsubsection*{b)}
Since
$$\sigma(q)\propto \frac{1}{(1+a^2q^2)^2}$$
where $k$ is a constant, we can rearrange the expression as:
$$\frac{1}{\sqrt{\sigma(q)}}\propto (1+a^2q^2).$$
By plotting $\frac{1}{\sqrt{\sigma(q)}}$ against $q^2$, i.e., let $y$-coordinate and $x$-coodinate be $\frac{1}{\sqrt{\sigma(q)}}$ and $q^2$ respectively, $a$ can be easily determined by calculating the square root of the slope of the fitted line.

\subsubsection*{c)}
For a delta potential,
$$U(\mathbf r)\propto \delta^{(3)}(\mathbf r)..$$
We can write
$$V_{si}\propto \int d\mathbf r\  e^{-i\mathbf q\mathbf r}\cdot\delta^{(3)}(\mathbf r).
$$
By applying the sifting property at $\mathbf r=\mathbf 0,$
$$V_{si}\propto  e^{-i\mathbf q\mathbf0}= \text{const.}
$$
It is found that scattering cross-section $V_{si}$ is independent of $q$. Thus,
$$\sigma(q)\propto |V_{si}|^2=\text{const.}
$$
\subsubsection*{d)}
The integral over all space for the specific Yukawa potential can be computed by changing to spherical coordinates:
\begin{align}
\int d\mathbf r\ \frac{e^{-r/a}}{4\pi a^2r}
&= \int_0^\infty dr \int_0^\pi d\theta \int_0^{2\pi} d\phi\  r^2\sin\theta\frac{e^{-r/a}}{4\pi a^2r}\\
&= \int_0^\infty dr\ 4\pi r^2\cdot \frac{e^{-r/a}}{4\pi a^2r}\\
&= \frac{1}{a^2}\int_0^\infty dr\  r\cdot e^{-r/a}\\
&= \frac{1}{a^2}\Big[-are^{-r/a}-a^2e^{-r/a} \Big]^{r=+\infty}_{r=0}\\
&= 1,
\end{align}
indicating the function is normalised.
\\
Consider the function at $a\to 0$, as the exponenatial decay is faster than any power of $a$,
$$\forall r>0,\ \lim_{a\to 0} \frac{e^{-r/a}}{4\pi a^2r}=0.$$
For any fixed $R>0$, the mass outside a sphere of radius $R$ vanishes as $a\to 0$:
\begin{align}
\int_{|\mathbf r|>R} d\mathbf r\ \frac{e^{-r/a}}{4\pi a^2r}
&= \int_R^\infty dr\ \frac{1}{a^2}r e^{-r/a}
= \int_{R/a}^\infty dx\ x e^{-x}\xrightarrow[a\to 0]{}0,
\end{align}
Hence the function becomes localised at $\mathbf r=\mathbf 0$ while remaining normalised, so it is delta-like and
$$\lim_{a\to 0}\frac{e^{-r/a}}{4\pi a^2r}\equiv \delta^{(3)}(\mathbf r).$$
\newpage
\section{2. Scaling Analysis}
\subsection{2.1 Quantum mechanics}
\subsubsection*{a)}
We consider a hydrogen atom where an electron of mass $m$ and charge $-e$ is bound by a proton of charge $+e$ at a confined Bohr radius $r$. The total energy of the electron can be determined by the sum of kinetic energy and potential energy. 

Applying Heisenberg uncertainty principle, the scale of the momentum of the electron can be expressed as
$$p\sim \frac{\hbar}{r}.$$
And the kinetic energy scale is
$$T=\frac{p^2}{2m}\sim \frac{\hbar^2}{2mr^2}.$$
The potential energy experienced by the electron is
$$V\sim-\frac{e^2}{4\pi\varepsilon_0r}.$$
Therefore, the total energy scale is
$$E=T+V\sim \frac{\hbar^2}{2mr^2}-\frac{e^2}{4\pi\varepsilon_0r}.$$
The first derivation of the steady state must vanish:
$$\frac{dE}{dr}\sim-\frac{\hbar^2}{mr^3}+\frac{e^2}{4\pi\varepsilon_0r^2}=0,$$
which yields
$$r\sim\frac{4\pi\varepsilon_0\hbar^2}{me^2}.$$
This expression coincides with the exact Bohr radius:
$$a_0=\frac{4\pi\varepsilon_0\hbar^2}{me^2}.$$
The velocity scale can be determined from the momentum:
$$v=\frac{p}{m}\sim\frac{\hbar}{mr}.$$
Substituting $r=a_0$ gives
$$v\sim\frac{e^2}{4\pi\varepsilon_0\hbar}.$$
This expression also coincides with the exact term:
$$v=\frac{e^2}{4\pi\varepsilon_0\hbar}.$$
Therefore, the exact expression of total energy at Bohr radius is
\begin{align}
E(a_0)
&= T(a_0)+V(a_0)\\
&= \frac{1}{2}mv^2-\frac{e^2}{4\pi\varepsilon_0r}\\
&=\frac{1}{2}m\Big(\frac{e^2}{4\pi\varepsilon_0\hbar}\Big)^2-\frac{e^2}{4\pi\varepsilon_0\frac{4\pi\varepsilon_0\hbar^2}{me^2}}\\
&=-\frac{1}{2}\frac{me^4}{(4\pi\varepsilon_0\hbar)^2}.
\end{align}
\subsubsection*{b)}
We consider an electron of mass $m$ and charge $-e$ at a 1D square barrier of height $U$ with energy $E<U$. The Schrodinger equation of the electron can be written as
$$-\frac{\hbar^2}{2m}\frac{d^2\psi}{dx^2}+U\psi=E\psi.$$
The tunneling \textit{probability} decays exponentially and scales as
$$P(\text{transmission})\sim e^{-2ak},$$
where $k=1/\lambda$ ($[k]=\text{m}^{-1}$) and $\lambda$ is the tunneling decay length.

The energy deficit can be defined as $E_{\text{def}}=U-E$.

The local decay of the wavefunction can only depend on the particle mass $m$ ($[m]=\text{kg}$), the energy deficit $E_{\text{def}}$ ($[E_{\text{def}}]=\text{kg m}^2\text{s}^{-2}$) , and Planck’s constant $\hbar$ ($[\hbar]=\text{kg m}^2\text{s}^{-1}$). A dimensional analysis can then be carried out:
$$k\sim m^{\alpha}E_{\text{def}}^\beta\hbar^\gamma,$$
where $\alpha$, $\beta$ and $\gamma$ are constant.
\begin{align}
[k]
&\sim \text{kg}^{\alpha}\cdot (\text{kg m}^2\text{s}^{-2})^\beta\cdot(\text{kg m}^2\text{s}^{-1})^\gamma\\
&\sim \text{kg}^{\alpha+\beta+\gamma} \cdot \text{m}^{2\beta+2\gamma}\cdot \text{s}^{-2\beta-\gamma}
\end{align}
Therefore, solving this set of equations
\begin{align}
\left\{
\begin{aligned}
& \alpha+\beta+\gamma=0 \\
& 2\beta+2\gamma=-1 \\
& -2\beta-\gamma=0
\end{aligned}
\right.
\end{align}
gives $\alpha=\beta=1/2,\ \gamma=-1$. Hence we can write $k$ as
$$k\sim\frac{\sqrt{mE_{\text{def}}}}{\hbar}.$$

Considering the transmission probability, the scaling result obtained can be interpreted physically as follows:

I. The decay constant scales as $k\propto \sqrt{m}$, indicating that heavier particles experience a stronger exponential suppression under the barrier. Consequently, heavier particles tunnel less efficiently than lighter ones.

II. The decay constant increases with the energy deficit $E_{\text{def}}$. As the particle energy approaches the barrier height ($E_{\text{def}}\to 0^+$), $k\to 0$ and the exponential suppression weakens, enhancing tunnelling. For $E_{\text{def}}<0$, the region becomes classically allowed, the decay constant becomes imaginary, and the wavefunction is oscillatory rather than evanescent, so tunnelling ceases to be the relevant process.

III. Planck’s constant $\hbar$ controls the degree of quantum delocalisation. Since $k\propto \hbar^{-1}$, increasing $\hbar$ reduces the decay rate and enhances tunnelling. In the classical limit $\hbar\to 0$, $k\to\infty$ and the wavefunction cannot penetrate the barrier, demonstrating that tunnelling is a purely quantum effect.
\newpage
\subsection{2.2 Transport phenomena in electrolytes}
\subsubsection*{a)}
The current density is defined as $j=I/A$. In a time interval $\Delta t$, ions with number density $n$, charge $ze$ and drift velocity $v$ cross an area $A$ from a slab of thickness $v\Delta t$. The number of ions crossing is $nA v\Delta t$, carrying total charge $\Delta Q = nA v\Delta t \cdot ze$. Hence
\[
j=\frac{1}{A}\frac{\Delta Q}{\Delta t}=nzev.
\]
\subsubsection*{b)}
Since $v=\mu E$,
\begin{align}
[v]=[\mu E]&\Rightarrow \text{m s}^{-1}=[\mu]\cdot  \text{kg m}\ \text{s}^{-3} \text{A}^{-1}
\\
&\Rightarrow [\mu] = \text{kg}^{-1} \text{s}^{2} \text{A}
\end{align}
\subsubsection*{c)}
Since $j=nzev$ and $v=\mu E$,
$$j=nze\mu E.$$
With $j=\sigma E$, the relation between conductivity and mobility can be written as
$$\sigma=nze\mu.$$
The dimensionality of conductivity can determined as follows:
\begin{align}
[\sigma]
&= [nze\mu]\\
&= \text{m}^{-3}\cdot\text{A s}\cdot\text{kg}^{-1} \text{s}^{2} \text{A}\\
&= \text{kg}^{-1} \text{m}^{-3} \text{s}^{3} \text{A}^2.
\end{align}

\newpage
\subsection{2.3 Chemical kinetics}
\subsubsection*{a)}
For a second-order bimolecular reaction with rate equation as
$$\frac{dA}{dt}=-KA^2,$$
where $A$ represents volume concentration, the dimensionality of rate constant can be determined as follows:
\begin{align}
[K]
&= [-A^{-2}\frac{dA}{dt}]\\
&= \text{m}^{6}\cdot \text{m}^{-3}\text{s}^{-1}\\
&= \text{m}^{3}\text{s}^{-1}.
\end{align}

\subsubsection*{b)}
The mean-free path length $\lambda$ can be estimated as
$$\lambda=\Big(\frac{V}{N}\Big)^\alpha\frac{1}{\sigma^\beta},$$
where $\alpha$ and $\beta$ are constants.
Dimensional analysis gives
$$\text{m}=\text{m}^{3\alpha-2\beta}.$$
Therefore the simplest solution for the assumption is $\alpha=\beta=1$, and 
$$\lambda=\frac{V}{\sigma N}.$$
The relative speed of each collision of molecules $v_c$ is then
$$v_c=\frac{\lambda}{\tau}=\frac{V}{\tau\sigma N}.$$
Consider the dimensionality, the rate constant can be estimated by multiplying the speed of collision by the cross-section of the molecules:
$$K\sim v_c\sigma=\frac{V}{\tau N}.$$
This estimate is physically sensible because a bimolecular reaction rate is controlled by the frequency of molecular encounters.
\subsubsection*{c)}
For gas molecules, applying equipartition for translational motion gives
$$\frac{1}{2}mv^2=\frac{3}{2}k_BT \Rightarrow v=\sqrt{\frac{3k_BT}{m}}.$$
With the conclusion in 2.3 b) and that $v_c\sim v$, we can write the scaling expressions for $\tau$ and $K$ as:
\begin{align}
\left\{
\begin{aligned}
& \tau\sim\frac{\lambda}{v}= \frac{V}{\sigma N}\sqrt{\frac{m}{3k_BT}}\\
& K \sim v_c\sigma=\sqrt{\frac{3k_BT}{m}} \sigma \\
\end{aligned}
\right.
\end{align}
\newpage
\section{3. Solving Equations}
\subsection{3.1 Using Lagrange and Newton methods}
\subsubsection*{a)}
For the equation $a\phi+b\phi^3-h=0$, we can rewrite it as
$$h=a\phi+b\phi^3.$$

For the case of large $h$, the solution $\phi$ is expected to be large, so the cubic term dominates over the linear term. We therefore factor out the leading cubic behaviour by writing
$$\phi(h)=\left(\frac{h}{b}\right)^{1/3}y,$$
where $y\to 1$ as $h\to\infty$.

Substituting this form into the equation gives
$$h=b\left(\frac{h}{b}\right)y^3+a\left(\frac{h}{b}\right)^{1/3}y
=hy^3+a\left(\frac{h}{b}\right)^{1/3}y.$$
Dividing by $h$ on both sides yields
$$y^3-1+\delta\,y=0,\ \text{where}\ \delta=\frac{a}{b^{1/3}h^{2/3}}.$$


Since $h$ is large, $\delta\ll1$, and we can expand the solution as a series in $\delta$ as
$$y=1+c_1\delta+c_2\delta^2+c_3\delta^3+O(\delta^4),$$
where $c_1$, $c_2$, $c_3$ are constants.

Substituting this expansion into the equation and equating coefficients order by order in $\delta$ gives
$$c_1=-\frac{1}{3},\ c_2=0,\ c_3=\frac{1}{81}.$$

Hence,
$$y=1-\frac{1}{3}\delta+\frac{1}{81}\delta^3+O(\delta^4).$$
Substituting back for $\phi$, we obtain the large-$h$ asymptotic solution
$$\phi(h)
=\frac{1}{b^{1/3}} h^{1/3}
-\frac{a}{3b^{2/3}}\,h^{-1/3}
+\frac{a^3}{81b^{4/3}}\,h^{-5/3}
+O(h^{-3}).$$
The perturbation theory from the lecture gives the result as
$$\phi\approx\Big(\frac{h}{b}\Big)^{1/3},$$
which is exactly the first term of the solution here.
\subsubsection*{b)}
For the equation 
$$e^{-ax}=bx,$$
with \textit{common sense}, if $b\gg a$, the linear function bx rises very steeply from the origin, while the exponential $e^{-ax}$ varies slowly for small $x$. The intersection therefore occurs at a small value of $x$, for which $ax\ll1$ and the exponential remains close to unity, $e^{-ax}\approx1$. The equation then reduces to $bx\approx1$, yielding the estimate
$$x\approx\frac{1}{b}.$$

On the other hand, if $a\gg b$, the RHS is small for any reasonable value of $x$. Therefore, the LHS must also be small, which requires $ax\gg 1$. This implies that the solution occurs at a value of $x$ slightly larger than the characteristic scale $1/a$. In this regime, the linear term has magnitude $bx\sim b/a$, so consistency of the equation requires
$$e^{-ax}\sim \frac{b}{a}.$$

Taking logarithms then yields
$ax \sim \ln\!\left(\frac{a}{b}\right),$
and hence
$$x \approx \frac{1}{a}\ln(\frac{a}{b}).$$

With Newton's method, we create a new function $f(x)$ such that
$$f(x)=e^{-ax}-bx,$$
where the derivative of $f(x)$ is
$$f'(x)=-ae^{-ax}-b.$$
If $b\gg a$, $x_0\approx1/b$, and
\begin{align}
x_1
&= x_0-\frac{f(x_0)}{f'(x_0)}\\
&= \frac{1}{b}+\frac{e^{-a/b}-1}{ae^{-a/b}+b}.
\end{align}
If $a\gg b$, $x_0\approx1/a\ln(a/b)$, and 
\begin{align}
x_1
&= x_0-\frac{f(x_0)}{f'(x_0)}\\
&= \frac{1}{a}\ln(\frac{a}{b})-\frac{e^{-a\cdot\frac{1}{a}\ln(\frac{a}{b})}-b\cdot\frac{1}{a}\ln(\frac{a}{b})}{-ae^{-a\cdot\frac{1}{a}\ln(\frac{a}{b})}-b}\\
&= \frac{1}{a}\ln(\frac{a}{b})-\frac{e^{-\ln(\frac{a}{b})}-\frac{b}{a}\ln(\frac{a}{b})}{-ae^{-\ln(\frac{a}{b})}-b}\\
&= \frac{1}{a}\ln(\frac{a}{b})-\frac{\frac{b}{a}(1-\ln(\frac{a}{b}))}{-2b}\\
&= \frac{\ln(\frac{a}{b})}{2a}+\frac{1}{2a}.
\end{align}
For the case of $b\gg a$, the first Newton correction is close to 0 as $(e^{-a/b}-1)\approx 0$, and the result from \textit{common sense} and from Newton's method (in the first-order) will be nearly identical. For the case of $a\gg b$, the discrepancy is much greater, as $x_1/x_0\approx1/2$. Newton's method provides a much more accurate estimate here.
\newpage
\section{4. Evaluating Integrals}
\subsection{4.1 Using Laplace method}
Let $f(x)=x^pe^{-\sqrt x}$, then the integral can be written as 
$$I(p)=\int^{+\infty}_0dx\ f(x).$$
We can rewrite the integral:
\begin{align}
I(p)
&= \int^{+\infty}_0dx\ e^{\ln f(x)}\\
&= \int^{+\infty}_0dx\ e^{p[\frac{1}{p}\ln f(x)]}\\
&= \int^{+\infty}_0dx\ e^{pH(x)},
\end{align}
where $H(x)=\frac{1}{p}\ln f(x)=\frac{1}{p}\ln(x^pe^{-\sqrt x})$.
The first and second derivative of $H(x)$ are
\begin{align}
\left\{
\begin{aligned}
& H'(x)=-\frac{1}{2p}x^{-\frac{1}{2}}+x^{-1}
\\
& H''(x)=\frac{1}{4p}x^{-\frac{3}{2}}-x^{-2}
\end{aligned}
\right.
\end{align}
The $H(x)$ has maximum at $H'(x)=0$, and
$$-\frac{1}{2p}x^{-\frac{1}{2}}+x^{-1}=0\Rightarrow x^*=4p^2.$$
Since $p\gg1$, $x^*$ must also be much greater than 1, and therefore
\begin{align}
I(p)
&\approx \int^{+\infty}_{-\infty}dx\ e^{pH(x)}\\
&\approx e^{pH(x^*)}\int^{+\infty}_{-\infty}dx\ e^{-\frac{1}{2}p\big(-H''(x^*)\big)(x-x^*)^2}\\
&= e^{p\frac{1}{p}\ln((4p^2)^pe^{-\sqrt {4p^2}})}\int^{+\infty}_{-\infty}dx\ e^{\frac{1}{2}p\big(\frac{1}{4p}(4p^2)^{-\frac{3}{2}}-(4p^2)^{-2}\big)(x-4p^2)^2}\\
&= 4^pp^{2p}e^{-2p}\int^{+\infty}_{-\infty}dx\ e^{\frac{1}{2}p\big(\frac{1}{4p}\frac{1}{8p^3}-\frac{1}{16p^4}\big)(x-4p^2)^2}\\
&= 4^pp^{2p}e^{-2p}\int^{+\infty}_{-\infty}dx\ e^{-\frac{1}{64p^3}(x-4p^2)^2}.
\end{align}
It can be noticed that the integral has a Gaussian shape, and hence
$$\int_{-\infty}^{+\infty}dx\ e^{-\frac{1}{64p^3}(x-4p^2)^2}
=\sqrt{2\pi}\,\sqrt{32p^3}
=\sqrt{64\pi p^3}
=8\sqrt{\pi}p^{3/2}.$$
Therefore,
\begin{align}
I(p)
&\approx 4^pp^{2p}e^{-2p}8\sqrt{\pi}p^{3/2}\\
&= 2^{3+2p}\sqrt{\pi}\ p^{2p+3/2}e^{-2p}.
\end{align}
\newpage
\subsection{4.2 Countour integration}
\subsubsection*{a)}
Applying $\cos px=\frac{1}{2}(e^{ipx}+e^{-ipx})$, the integral becomes
\begin{align}
F(p,b)
&= \int^{+\infty}_{0}dx\ \frac{\frac{1}{2}(e^{ipx}+e^{-ipx})}{(x^2+b^2)^2}\\
&= \frac{1}{2}\Big( \int^{+\infty}_{0}dx\ \frac{e^{ipx}}{(x^2+b^2)^2}+\int^{+\infty}_{0}dx\ \frac{e^{-ipx}}{(x^2+b^2)^2}\Big)\\
&= \frac{1}{2}\Big( \int^{+\infty}_{0}dx\ \frac{e^{ipx}}{(x^2+b^2)^2}-\int^{+\infty}_{0}d(-x)\ \frac{e^{ip(-x)}}{((-x)^2+b^2)^2}\Big)\\
&= \frac{1}{2}\Big( \int^{+\infty}_{0}dx\ \frac{e^{ipx}}{(x^2+b^2)^2}-\int^{-\infty}_{0}dx\ \frac{e^{ipx}}{(x^2+b^2)^2}\Big)\\
&= \frac{1}{2}\Big( \int^{+\infty}_{0}dx\ \frac{e^{ipx}}{(x^2+b^2)^2}+\int^{0}_{-\infty}dx\ \frac{e^{ipx}}{(x^2+b^2)^2}\Big)\\
&= \frac{1}{2}\int^{+\infty}_{-\infty}dx\ \frac{e^{ipx}}{(x^2+b^2)^2}.
\end{align}
Define 
$$f(z)=\frac{e^{ipz}}{(z^2+b^2)^2},$$
then the residue theorem can be applied to $f(z)$ where the poles occur when the denominator is 0, i.e.
$$(z^2+b^2)^2=0\Rightarrow z_0=\pm ib,$$
both of which are second-order poles.

Since $\cos px$ is even, here we assume $p>0$, which does not affect the calculations.
For $p>0$, the contour is closed in the upper half-plane since $e^{ipz}$ decays exponentially as $\Im (z)\to+\infty$, so only the pole at $z_0=ib$ contributes.
We can compute the residue of the pole as
\begin{align}
\text{Res} f(z)
&= \lim_{z\to z_0}\frac{d}{dz}\Big[(z-ib)^2f(z)\Big]\\
&= \lim_{z\to z_0}\frac{d}{dz}\Big[(z-ib)^2\frac{e^{ipz}}{(z^2+b^2)^2}\Big]\\
&= \lim_{z\to z_0}\frac{d}{dz}\Big[(z-ib)^2\frac{e^{ipz}}{(z+ib)^2(z-ib)^2}\Big]\\
&= \lim_{z\to z_0}\frac{d}{dz}\Big[\frac{e^{ipz}}{(z+ib)^2}\Big]\\
&= \lim_{z\to z_0}\frac{ipe^{ipz}}{(z+ib)^2}-\frac{2e^{ipz}}{(z+ib)^3}.
\end{align}
At the upper contour where $z_0=ib$,
\begin{align}
\text{Res} f(z)
&= \frac{ipe^{ip\cdot ib}}{(2ib)^2}-\frac{2e^{ip\cdot ib}}{(2ib)^3}\\
&= \frac{ipe^{-pb}}{-4b^2}-\frac{2e^{-pb}}{-8ib^3}\\
&= \frac{-ie^{-pb}}{4b^2}(\frac{1}{b}+p)
\end{align}
Then the integral can be evaluated as 
\begin{align}
F(p,b)
&= \frac{1}{2}2\pi i\cdot Res\ f(ib)\\
&= \pi i\cdot \frac{-ie^{-pb}}{4b^2}(\frac{1}{b}+p)\\
&= \frac{\pi e^{-pb}}{4b^2}(\frac{1}{b}+p).
\end{align}
For generality of the sign of $p$, we can rewrite the result as
$$F(p,b)=\frac{\pi e^{-|p|b}}{4b^2}(\frac{1}{b}+|p|).$$
Note that the final result of the integral is real.
\subsubsection*{b)}
The result of $I(p,b)$ is 
$$I(p,b)=\frac{\pi}{2b}e^{-pb}.$$

We can notice that differentiate $I(p,b)$ with respect to $b$ can yield an expression similar to $F(b)$:
$$\frac{\partial I(p,b)}{\partial b}=\frac{\partial}{\partial b}\int^{+\infty}_{0}dx\ \frac{\cos px}{(x^2+b^2)}=\int^{+\infty}_{0}dx\ \frac{-2b\cos px}{(x^2+b^2)^2}=-2bF(p,b).$$
Substituting $F(p,b)$ gives
$$\frac{\partial I(p,b)}{\partial b}=\frac{-\pi e^{-pb}}{2b^2}(1+pb)$$
This partial derivative can also be directly calculated:
\begin{align}
\frac{\partial I(p,b)}{\partial b}
&= \frac{\partial }{\partial b}\frac{\pi}{2b}e^{-pb}\\
&=\frac{\pi}{2}(-\frac{e^{-pb}}{b^2}+\frac{-pe^{-pb}}{b} \\
&= \frac{-\pi e^{-pb} }{2b^{2}}(1+pb),
\end{align}
which reproduces the result exactly, and thus validating the contour result.


\subsubsection*{c)}
Substituting $U(r)$ into the expression of $\tilde U(q)$ derived in question 1 gives
$$\tilde U(q)=\frac{4\pi}{q}\int^{+\infty}_{0}dr\ \frac{r\sin (qr)}{\cosh(r/a)}.$$
Let $x=r/a$, then $dr=a\ dx$, and 
$$\tilde U(q)=\frac{4\pi a^2}{q}\int^{+\infty}_0 dx\ \frac{x\sin (qax)}{\cosh x}.$$
To evaluate this integral, we first consider the auxiliary integral
$$I(\alpha)=\int_0^{+\infty} dx\ \frac{\cos(\alpha x)}{\cosh x},
$$
where $\alpha=qa$. The required integral can then be obtained by differentiation, since
$$\int^{+\infty}_0 dx\ \frac{x\sin(\alpha x)}{\cosh x}
=
-\frac{dI}{d\alpha}.$$

Define
$$f(z)=\frac{e^{i\alpha z}}{\cosh z}.$$
The poles of $f(z)$ occur when denominator $\cosh z=0$. Using the exponential definition,
$$\cosh z=\frac{1}{2}(e^z+e^{-z})=0 \Rightarrow e^{2z}=-1,$$
With Euler's identity, we obtain
$$e^{2z}=e^{i(\pi+2k\pi)},$$
where $k\in Z$, which gives
$$z=\frac{i}{2}(\pi+2k\pi).$$

In the simple region of $0<\Im (z)<\pi$, we have one simple pole at $z_0=i\pi/2$. Expanding $\cosh z$ about $z_0$ gives
$$\cosh z=\cosh(z_0)+\sinh(z_0)(z-z_0)+O\!\left((z-z_0)^2\right),$$
and since $\sinh(i\pi/2)=i$, we obtain
\begin{align}
\text{Res} f(z)
&=\lim_{z\to z_0}\Big[(z-z_0)\frac{e^{i\alpha z_0}}{\cosh(z_0)+\sinh(z_0)(z-z_0)+O\!\left((z-z_0)^2\right)}\Big]\\
&= \lim_{z\to z_0}\Big[(z-z_0)\frac{e^{i\alpha z_0}}{i(z-z_0)+O\!\left((z-z_0)^2\right)}\Big]\\
&= -ie^{-\alpha\pi/2}.
\end{align}

Using a rectangular contour of height $i\pi$, the contributions from the vertical sides vanish as the width tends to infinity. Moreover, since $\cosh(z+i\pi)=-\cosh z$, the contributions from the top and bottom edges combine to give
$$\oint dz\ f(z)= 2\int^{+\infty}_{-\infty}dx\ \frac{e^{i\alpha x}}{\cosh x}.$$

By the residue theorem,
$$\oint dz\ f(z) = 2\pi i\,\operatorname{Res} f(z_0) = 2\pi e^{-\alpha\pi/2}.$$
Hence
$$\int^{+\infty}_{-\infty}dx\ \frac{e^{i\alpha x}}{\cosh x} = \pi e^{-\alpha\pi/2}.$$

Using Euler’s formula $e^{i\alpha x}=\cos(\alpha x)+i\sin(\alpha x)$, we take the real part of the integral. Since $\sin(\alpha x)$ is odd while $\cosh x$ is even, the imaginary part vanishes over $(-\infty, +\infty)$. The remaining integrand is even, so the integral over $(-\infty,+\infty)$ is twice that over $(0,+\infty)$, and therefore
$$I(\alpha)=\int_0^{+\infty} dx\ \frac{\cos(\alpha x)}{\cosh x} =\frac{\pi}{2} e^{-\alpha\pi/2}.$$
Note that since $e^{-\alpha\pi/2}=\text{sech}(\pi\alpha/2)$, we have
$$I(\alpha) =\frac{\pi}{2}\,\text{sech}\ (\frac{\pi\alpha}{2}).$$
Differentiating $I(\alpha)$ gives
\begin{align}
\int_0^{+\infty} dx\ \frac{x\sin(\alpha x)}{\cosh x}\,dx
&=-\frac{dI}{d\alpha}\\
&= \frac{\pi^2}{4}\ \text{sech}(\frac{\pi\alpha}{2})\tanh(\frac{\pi\alpha}{2}).
\end{align}
Substituting $\alpha=qa$ and returning to $\tilde U(q)$, we obtain
$$\tilde U(q)=\frac{\pi^2 a^2}{q}
\text{sech}(\frac{\pi aq}{2})
\tanh(\frac{\pi aq}{2}).$$

\end{document}
